\documentclass[a4paper,oneside]{article}

\usepackage[linesnumbered,vlined,boxed]{algorithm2e}
\SetAlgorithmName{Behaviour}{behaviour}{List of Behaviours}

\usepackage{minitoc}
\setcounter{secttocdepth}{5}
\setcounter{tocdepth}{1}

\usepackage{graphicx}
\usepackage{epstopdf}
\usepackage[export]{adjustbox}
\usepackage[titletoc,title,header,page]{appendix}

\author{Philip Hale}
\title{CS3017 Assessment: A Power-Law Peer-to-Peer System}

\begin{document}

\begin{titlepage}
  \maketitle
  %\vfill
  \begin{abstract}
  A peer-to-peer system consisting of Super Peers and Ordinary Peers, operating
  in a power-law topology. The peers interact to exchange files using the
  document routing P2P model. The application is implemented in the Java Agent
  DEvelopment Framework (JADE), and ships with a default configuration that
  simulates `Scenario 1', where one peer has a file which all other peers want.
  See Appendix 1 For details on configuration and running the simulation.
  \end{abstract}
  \dosecttoc
  \tableofcontents
\end{titlepage}

\section{Peer Behaviour}

The system consists of three types of agent: Super Peers, Ordinary Peers and a
single Host Cache.  The Host Cache is implemented as an agent in order to unify
the connection behaviour between elements of the system.  In other words, peers
can connect to the Host Cache using the same mechanisms that are required to
connect to each other.

It can be helpful to think of the identity of an agent as defined by the kinds
of behaviours it can support. What we mean here by a behaviour is the ability to
respond in some way to a particular situation or event.  Some behaviours
continually monitor the state of the agent in the network, whereas others
are triggered by incoming messages from other agents.

The process of specifying and identifying different messages is explained in
another section.

Super Peers and Ordinary Peers implement a common set of behaviours. For
the sake of simplicity, the designation of peers into Super and Ordinary sets
will occur during initialization of the system.

\subsection{Host Cache}

The Host Cache is the entry-point for the network with two responsibilities.
Firstly, to maintain a list of connected peers. Secondly, to supply peers with a
list of Super Peers which they can attempt to connect to

Theses behaviours can be summarised by the following piece of pseudocode. It
will run when the Host Cache receives a message identified as `Neighbours
Request'.

\subsubsection{Receive `Neighbours Request`}

\begin{algorithm}[H]
  \SetKwInOut{Input}{input}
  \SetKwFunction{ReplyFn}{reply}
  \Input{Message Name: `Neighbours Request' \\ Message Sender: $peer$}
  \KwData{$peerList$: A hash of peers with value true if peer is super}
  \If{peer list doesn't contain $peer$}{
    add $peer$ to $peerList$\;
  }
  \For{$peer \in peerList$}{
    \If{$peerList[peer] = true$}{
      $neighbours \leftarrow peer$\;
    }
  }
  \ReplyFn{`Neighbours Response', neighbours}
\caption{Receive `Neighbours Request'}
\end{algorithm}

\subsection{Peers}

Here we will explain the different types of behaviours implemented by both Super
and Ordinary Peers. Some of these behaviours are shared between all peers, but
their effects will differ depending on whether the peer is Super or Ordinary.
This is a design decision which compromises simplicity of the described
behaviours with simplicity of the system as a whole.\footnote{The idea is
analogous to the contradiction between two rules of software engineering: The
Law of Demeter which aims to minimise method chaining, and the idea of class
cohesion which gives an object well-bounded behaviour.}

This behaviour is responsible for maintaining a list of Super Peers that the
peer can connect to.  One consequence of this design is that the peers will
always have a list of potential peers to connect to, even when they already have
their maximum number of connected peers. Additionally, $knownPeers \cap
connectedPeers \neq \emptyset$.

\begin{algorithm}[H]
  \SetKwInOut{Input}{input}
  \SetKwFunction{SendFn}{send}
  \KwData{$knownPeers$: A list of neighbours known but not connected to the
  peer}
  \KwData{$hostCache$: Address of the Host Cache}
  \If{$knownPeers = \emptyset$}{
    \SendFn{hostCache, `Neighbours Request'}
  }
\caption{Send `Neighbours Request'}
\end{algorithm}

\begin{algorithm}[H]
  \SetKwInOut{Input}{input}
  \Input{Message Name: `Neighbours Response' \\ Neighbours: $peers$ \\ Message
  Sender: $sender$}
  \KwData{$knownPeers$: A list of neighbours known but not connected to the
  peer}
  \KwData{$hostCache$: Address of the Host Cache}
  \If{$sender = hostCache$}{
    $knownPeers \cap  peers$\;
  }
\caption{Receive `Neighbours Response'}
\end{algorithm}

\begin{algorithm}[H]
  \SetKwInOut{Input}{input}
  \SetKwFunction{SendFn}{send}
  \SetKwFunction{LenFn}{length}
  \KwData{$knownPeers$: A list of neighbours known but not connected to the
  peer}
  \KwData{$connectPendingPeers$: Peers that have been sent a connection request}
  \KwData{$minPeers$: Required number of connected peers.}
  \If{$knownPeers \neq \emptyset $ {and} \LenFn{$connectPendingPeers \cap
  connectedPeers$}$ < minPeers$}{
    \SendFn{$ \exists p \in knownPeers$, `Connect Request'}
  }
\caption{Send `Connect Request'}
\end{algorithm}









\subsection{Message Definitions}

\begin{itemize}
  \item{Neighbours Request}
  \item{Neighbours Response}
  \item{Search Request}
  \item{Search Response}
  \item{File Request}
  \item{File Response}
  \item{File List}
  \item{Connect Request}
  \item{Connect Response}
\end{itemize}









\section{Protocols}

\secttoc

The following abbreviations are used:

\begin{description}
  \item[OPeer] An Ordinary Peer
  \item[SPeer] A Super Peer
  \item[Req.] Send a request message
\end{description}

\subsection{Joining the network}

This involves contacting the Host Cache for the IDs of Super Peers, and
requesting a connection.  The connection can either be accepted or denied. If
the connection is accepted, the peer sends their list of shared files to the
connected Super Peer.

\includegraphics{protocol_connect.eps}

\subsection{Send file-list}

Only ordinary Peers send a file list.

\includegraphics{protocol_send_file_list.eps}

\subsection{Searching for Files}

In the following example, there are only two Super Peers in the network.  The
search request is forwarded throughout the network, and if at some point a Super
Peer is able to locate it, the location is passed back through the network to
the original requester.

\includegraphics{protocol_file_search.eps}

\subsection{Acquiring Files}

Once a file location has been determined, acquiring it is straightforward direct
communication between two peers.

\includegraphics{protocol_acquire_file.eps}

\section{Message Definitions}

\begin{itemize}
  \item{Neighbours Request}
  \item{Neighbours Response}
  \item{Search Request}
  \item{Search Response}
  \item{File Request}
  \item{File Response}
  \item{File List}
  \item{Connect Request}
  \item{Connect Response}
\end{itemize}




\begin{appendices}
  \section{Usage Instructions}

  \subsection{From the command line}
  To run the simulation from the command line:
  \begin{enumerate}
  \item{mvn clean compile exec:java}
  \end{enumerate}

  \subsection{Eclipse}
  To run the simulation in Eclipse:
  \begin{enumerate}
  \item{Extract the zip to a folder}
  \item{Using Eclipse, choose: File -> Import -> Maven -> Existing Maven
  Projects.}
  \item{Set Root Directory to be the extracted folder. It should identify the
  pom.xml file}
  \item{Next -> Finish.
  \item{Run -> Run Configurations -> Maven Build -> mvn_exec_java}
  \item{Run}
  \end{enumerate}

  The simulation will run indefinitely.  The output is a series of 

  \subsubsection{Deviations and limitations}
  Analysis of the behaviour of the system reveals certain areas which diverge in
  some way from the required behaviour.

  If a Super Peer is disconnected from the network, then there is no mechanism
  for its connected peers to be made aware of this fact.  Nonetheless, once a
  Peer has been told that the Super Peer is disconnected (either by a third part
  or by the peer periodically Pinging their current connections), the simulation
  would correctly handle the situation by adding more connections if necessary.
  Therefore, what remains is an extra Behaviour for all peers to periodically
  check if their connections are still connected.

  Instead of reporting to the super peer all the files it requests, Peers send
  search requests individually. The resulting behaviour is identical, although
  it means for peers with more than 1 Super Peer connection (i.e. Super Peers),
  each search request may go to a different connected Super Peer.

  One limitation that has surfaced despite following the spec is the potential
  for islands to form in the network. Since no guarantee is placed on which
  peers connect to which, it's possible for the network to form into two or more
  groups.  For example, take the case where Super Peers can have a maximum of 2
  peers connected to them. Imagine there are 2 Super peers connected to each
  other, leaving one connection space free for both.  This connection space
  might be occupied by an Ordinary Peer, for both Super Peers.  This set of 4
  peers is now unable to join any other parts of the network.  If one of these
  peers is the peer which has a file everyone else wants (in the case of
  Scenario 1), the rest of the network will be unable to retrieve this file.
  This problem can be made less likely by having a smaller number of Super Peers
  in the network (and adjusting the capacity of super peers accordingly).

  It is not possible to directly change the bandwidth of individual peers or
  groups of peers. This is largely because of difficulties in specifying
  individual properties for peers when there are a large number of peers being
  generated.  In order to simulate the effects of changing the bandwidth of
  peers, you can specify the proportion of peers that should be Super Peers.
  This is a probability; no guarantees are given to the actual number of Super
  Peers.

  \section{Configuration}

  The project can be configured in a variety of ways. For example, by changing
  the total number of peers or the proportion of Super Peers.

  The configuration supplied is set to create 500 Peers, of which about 125 are
  Super Peers. Each Super Peer is able to accept connection requests from up to
  12 Ordinary Peers. Ordinary Peers, as per the spec, always connect to 1 Super
  Peer.  Super peers on the other hand must connect to 3 other Super Peers. The
  Host Cache returns the addresses of 10 Super Peers when asked.

  The files shared and wanted by peers are configured to match Scenario 1 in the
  spec, where 1 peer has a file that everyone else wants.

  There are some additional 



  For a full list of options and how they change the simulation, see the
  commented P2P_PowerLaw.properties file.

\end{appendices}

\end{document}
