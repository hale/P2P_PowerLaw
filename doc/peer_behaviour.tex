\section{Peer Behaviour}

\secttoc

The system consists of three main types of agent: Super Peers, Ordinary Peers and a
single Host Cache.

It can be helpful to think of the identity of an agent as defined by the kinds
of behaviours it can support. A behaviour is the ability to respond in some way
to a particular situation or event.  Some behaviours continually monitor the
state of the agent in the network, whereas others are triggered by incoming
messages from other agents.

Thinking of the agents in this way makes it easier to describe their
functionality.  The algorithms listed in this section describe isolated pieces
of functionality that agents implement.  behaviours under the heading `Peer'
are implemented by both Ordinary Peers and Super Peers.

It is also constructive to split the behaviours of the peers in this way, since
this is how the Peers must be described in JADE. The algorithms were developed
in advance of the implementation, and refined when problems were found with the
logical procedure.

Several simplifications have been made in these descriptions.  These are as
follows:

\begin{itemize}
\item{No distinction is made between the name of a file and the file itself.}
\item{The \verb!send()! function will send a named message to the specified
peer}
\item{The system is well-behaved: we do not account for messages to be sent
erroneously, or by the wrong agent.}
\item{For some of the `Send...' actions, it will be necessary to either
introduce a delay or store additional state about e.g. if the peer has already
made the request. This issue has been ignored for the purposes of this
section.}
\end{itemize}

In the following algorithms, the input for behaviours beginning with `Receive'
include a message name.  In effect this means ``the procedure is triggered by a
message with this name''.

\subsection{Host Cache}

The Host Cache is the entry-point for the network with two responsibilities.
Firstly, to maintain a list of connected peers. Secondly, to supply peers with a
list of Super Peers which they can attempt to connect to.

\subsubsection{Receive `Neighbours Request`}

Note: In this description, the Host Cache would return every Super Peer
registered on the network. In practice, we need to define a limit on the total
size of the response.

\begin{algorithm}[H]
  \SetKwInOut{Input}{input}
  \SetKwFunction{SendFn}{send}
  \Input{Sender: $peer$ \\ Message Name: `Neighbours Request' \\ $isSuper$ True
  if the peer is a Super Peer}
  \KwData{$peerList$: A hash of peers with value true if peer is super}

  \If{$peer \notin peerList$}{
    add $peer$,$isSuper$ to $peerList$\;
  }
  \For{$connectedPeer \in peerList$}{
    \If{$connectedPeer$ is super and $connectedPeer \neq peer$}{
      $neighbours \leftarrow neighbours \cup peer$\;
    }
  }
  \SendFn{peer, `Neighbours Response', neighbours}\;
\end{algorithm}

\subsection{All Peers}

Here we will explain the functionality common to both Super and Ordinary Peers.
Some of these behaviours are shared between all peers, but their effects will
differ depending on whether the peer is Super or Ordinary.  This is a design
decision which compromises simplicity of the described behaviours with
simplicity of the system as a whole.

\subsubsection{Send `Neighbours Request'}

The first message a peer will send: asks the Host Cache for a list of peers
which it can connect to in order to join the network. In addition, this message
is sent whenever the peer runs out of known peers, and still needs more
connections.

\begin{algorithm}[H]
  \SetKwFunction{SendFn}{send}
  \KwData{$knownPeers$: A list of neighbours known but not connected to the
  peer}
  \KwData{$hostCache$: Address of the Host Cache}
  \KwData{$isSuper$: True if the peer has sufficient bandwidth to be a Super
  Peer }

  \If{$knownPeers = \emptyset$ and $\neg hasRequestedPeers$}{
    \SendFn{hostCache, `Neighbours Request', isSuper}\;
  }
\end{algorithm}

\subsubsection{Receive `Neighbours Response'}

A reply to the `Neighbours Request' message. The message payload contains a
list of Super Peers which the peer can try and connect to.

\begin{algorithm}[H]
  \SetKwInOut{Input}{input}
  \Input{Message Name: `Neighbours Response' \\ Potential Connections: $peers$}
  \KwData{$knownPeers$: A list of Super Peers known but not connected to the
  peer}

  $knownPeers \leftarrow knownPeers \cup  peers$\;
\end{algorithm}

\subsubsection{Send `Connect Request'}

Asks a Super Peer to be a connection. The contacted Super Peer is placed in a
list while we wait for a response.

The peer will keep trying to connect to more peers until the number of
connected peers is greater than a given minimum. In order to limit how many
connection attempts are made at once, the count includes peers which we are
waiting on for a reply.

\begin{algorithm}[H]
  \SetKwFunction{SendFn}{send}
  \KwData{$knownPeers$: A list of Super Peers known but not connected to the peer}
  \KwData{$connectPendingPeers$: Super Peers that have been sent a connection request}
  \KwData{$connectedPeers$: Super Peers that this peer is connected to}
  \KwData{$minPeers$: Required number of connected peers.}

  \If{$knownPeers \neq \emptyset $ {and} $\|connectPendingPeers \cup
  connectedPeers\| < minPeers$}{
    \SendFn{$ \exists p \in knownPeers$, `Connect Request'}\;
  }
\end{algorithm}

\subsubsection{Receive `Connect Response'}

A connection response is either successful or unsuccessful.

Upon receipt of this response, the contacted Super Peer is removed from the
known peers list and the pending peers list. If the connection is successful,
the peer is added to the list of connected peers.

\begin{algorithm}[H]
  \SetKwInOut{Input}{input}
  \Input{Sender: $sender$ \\ Message Name: `Connect Response' \\ Connection
  Successful?: $connectSucceed$}
  \KwData{$knownPeers$: A list of Super Peers known but not connected to the peer}
  \KwData{$connectPendingPeers$: Super Peers that have been sent a connection request}
  \KwData{$connectedPeers$: Super Peers that this peer is connected to}

  $knownPeers \leftarrow knownPeers \setminus \{sender\}$\;
  $connectPendingPeers \leftarrow connectPendingPeers \setminus \{sender\}$\;
  \If{$connectSucceed = true$}{
    $connectedPeers \leftarrow connectedPeers \cup \{sender\}$\;
  }
\end{algorithm}

\subsubsection{Receive `File Request'}

Since our simulation does not send actual files, at this point we can simply
send the filename back to the message sender, to acknowledge a file transfer.

\begin{algorithm}[H]
  \SetKwInOut{Input}{input}
  \SetKwFunction{SendFn}{send}
  \Input{Sender: $peer$ \\ Message Name: `File Request' \\ File: $file$}
  \KwData{$sharedFiles$: List of files the peer wants but does not have}

  \SendFn{peer, `File Response', $file \in sharedFiles$}\;
\end{algorithm}

\subsubsection{Receive `File Response'}

This completes the procedure for acquiring files in the network.

\begin{algorithm}[H]
  \SetKwInOut{Input}{input}
  \SetKwFunction{SendFn}{send}
  \Input{Message Name: `File Response' \\ File: $file$ }
  \KwData{$sharedFiles$: List of files the peer wants but does not have}
  \KwData{$transferringFiles$: List of files that are waiting to be sent to this
  peer.}
  \KwData{$wantedFiles$: List of files the peer wants but doesn't have}

  $transferringFiles \leftarrow transferringFiles \setminus \{file\}$\;
  $wantedFiles \leftarrow wantedFiles \setminus \{file\}$\;
  $sharedFiles \leftarrow sharedFiles \cup \{file\}$\;
\end{algorithm}

\subsection{Ordinary Peers}

These behaviours are exclusively implemented by Ordinary Peers.

\subsubsection{Send `File List'}

An Ordinary Peer is required to submit its list of shared files to a connected
Super Peer.  This is so that other peers can search for their files.

Since this behaviour only applies to Ordinary Peers, we can make the simplifying
assumption that the list of connected peers only contains a single peer.

One further assumption is made.  We presume that the list of files that is sent
to the Super Peer is small enough to fit in a single message. In practice, this
is not likely to be the case.

\begin{algorithm}[H]
  \SetKwFunction{SendFn}{send}
  \KwData{$connectedPeer$: Super Peer that this peer is connected to}
  \KwData{$sharedFiles$: List of files the peer wants but does not have}

  \SendFn{connectedPeer, `File List', sharedFiles}\;
\end{algorithm}

\subsubsection{Receive `Search Response'}

When a peer receives a search result, the nature of the protocol means that this
must be a successful result. If there are no results found, then there is simply
no message returned.  The payload of the message includes the particular file
that was found and the ID of the peer who owns it.

\begin{algorithm}[H]
  \SetKwInOut{Input}{input}
  \SetKwFunction{SendFn}{send}
  \Input{Message Name: `Search Response' \\ Result: $file$ \\ Peer With File:
  $peer$}
  \KwData{$transferringFiles$: List of files that have been requested
  from another peer, and are waiting to be sent to this peer.}
  \KwData{$wantedFiles$: List of files the peer wants but doesn't have}

  $wantedFiles \leftarrow wantedFiles \setminus \{file\}$\;
  \SendFn{peer, `File Request', file}\;
  $transferringFiles \leftarrow transferringFiles \cup \{file\}$\;
\end{algorithm}

\subsubsection{Send `Search Request'}

For this demonstration, a peer will always and continually search for files they
want.

\begin{algorithm}[H]
  \SetKwFunction{SendFn}{send}
  \KwData{$connectedPeers$: Super Peers that this peer is connected to}
  \KwData{$wantedFiles$: List of files the peer wants but does not have}

  \If{$wantedFiles \neq \emptyset$ {and} $connectedPeers \neq \emptyset $}{
    \SendFn{$\exists p \in connectedPeers$, `Search Request', $\exists f \in
    wantedFiles$}\;
  }
\end{algorithm}

\subsection{Super Peers}

These behaviours are exclusively implemented by Super Peers.

\subsubsection{Receive `Connection Request'}
input: a request from a peer to join; output: connection response (y/n)

Super Peers maintain a lit of peers which are connect to them, and the files
they are sharing.  The first step in this interaction is for the Peer to
associate itself with the Super Peer.

\begin{algorithm}[H]
  \SetKwFunction{SendFn}{send}
  \SetKwInOut{Input}{input}
  \Input{Sender: $peer$ \\ Message Name: `Connection Request'}
  \KwData{$connectedOrdinaryPeers$: Ordinary Peers that this peer is connected to}
  \KwData{$ordinaryPeerLimit$: Maximum number of Ordinary Peers this Super Peer
  can connect to}

  \eIf{$\|connectedOrdinaryPeers\| < ordinaryPeerLimit$}{
    $connectedOrdinaryPeers \leftarrow connectedOrdinaryPeers \cup \{peer\}$\;
    \SendFn{peer, `Connect Response', true}\;
  }{
    \SendFn{peer, `Connect Response', false}\;
  }
\end{algorithm}


\subsubsection{Receive `File list'}

The Super Peer stores the list of files for each Ordinary Peer it is connected
to. This is so that it is able to handle search requests.

The files are stored as a hash that maps $fileName \rightarrow peerList$.

\begin{algorithm}[H]
  \SetKwInOut{Input}{input}
  \Input{Sender: $peer$ \\ Message Name: `File List' \\ Shared Files:
  $fileList$}
  \KwData{$connectedOrdinaryPeers$: Ordinary Peers this peer is connected to}
  \KwData{$ordinaryPeersFiles$: All shared files of connected Ordinary Peers}

  \For{$file \in fileList$}{
    $ordinaryPeersFiles[file] = ordinaryPeersFiles[file] \cup \{peer\}$\;
  }
\end{algorithm}

\subsubsection{Send `Search Request'}

Since a Super Peer also has a list of files, it makes sense to search these
first before searching network.

\begin{algorithm}[H]
  \SetKwFunction{SendFn}{send}
  \KwData{$wantedFiles$: List of files the peer wants but does not have}
  \KwData{$ordinaryPeersFiles$: All shared files of connected Ordinary Peers}

  \If{$wantedFiles \neq \emptyset$}{
    \eIf{$\exists file,peers \in ordinaryPeersFiles where file \in wantedFiles $}{
      `Search Response' behaviour, with input $\exists p \in peers$ and $file$\;
    }{
      inherit behaviour from ordinary peer\;
    }
  }
\end{algorithm}



\subsubsection{Receive `Search Response'}

Super Peers must behave slightly differently when receiving a Search Response
message. Unlike for Ordinary Peers, sometimes this response may not be for them.
This is because search results do not get delivered directly: they are passed
back through intermediary peers using the Document Routing system.

\begin{algorithm}[H]
  \SetKwInOut{Input}{input}
  \SetKwFunction{SendFn}{send}
  \Input{Message Name: `Search Response' \\ Result: $file$ \\ Peer With File:
  $peer$ \\ Previous Senders of the Search Response: $senderStack$}

  \uIf{$senderStack = \emptyset$}{
    inherit the behaviour of Ordinary Peer\;
  }{
    $superPeer \leftarrow senderStack.pop()$\;
    \SendFn{superPeer, 'Search Response', file, peer, senderStack}\;
  }
\end{algorithm}

\subsubsection{Receive `Search Request'}

When a Super Peer receives a Search Request for a file, it first checks its list
of files. If the file is found, it replies with a Search Response. 

If it is not found, it must pass on the message. If the $senderStack$ is empty
then the sender must be an ordinary peer, so the sender is added to it.  Next,
it adds itself to the $senderStack$ and passes on the message to $\exists peer
\in connectedOrdinaryPeers$.

The peer chosen to receive this message is picked based on the proximity of
their ID when compared to the file ID.  The details of this operation are not
explained in this section, but it introduces a requirement of the implemented
system for the constructed IDs to have the property of total ordering.

\begin{algorithm}[H]
  \SetKwFunction{SendFn}{send}
  \SetKwFunction{NearestFn}{nearest}
  \SetKwInOut{Input}{input}
  \Input{Sender: $peer$ \\ Message Name: `Search Request' \\ Requested File:
  $file$ \\ Previous Senders of the Search Response: $senderStack$}
  \KwData{$connectedOrdinaryPeers$: Ordinary Peers connected to this Super Peer}
  \KwData{$ordinaryPeersFiles$: All shared files of connected Ordinary Peers}
  \KwData{$self$: This peer}

  \eIf{$ordinaryPeersFiles$ contains $file$}{
    $peerWithFile \leftarrow \exists p \in ordinaryPeersFiles[file] $\;
    \eIf{$senderStack = \emptyset$}{
      \SendFn{peer, `Search Response', file, peerWithFile}
    }{
      \SendFn{$senderStack.pop()$, `Search Response', file, peerWithFile,
      senderStack}\;
    }
  }{
    \tcc{We don't have the file; forward request to neighbour}
    $nearestNeighbour \leftarrow$ \NearestFn{senderStack, file}\;
    $senderStack \leftarrow senderStack \cup \{self\}$\;
    \SendFn{nearestNeighbour, `Search Request', file, senderStack}\;
  }

\end{algorithm}
