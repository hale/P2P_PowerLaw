\section{Usage Instructions}

The project was created in IntelliJ IDEA, and a settings file has been created
for an Eclipse project. A pom.xml file has been provided, which can be used
to fetch dependencies, compile and launch the simulation regardless of
platform and IDE\@. The instructions below describe importing the project into
Eclipse as a Maven project - it may open in Eclipse without these steps, but
this should ensure that the dependencies are fetched.

For your information, the following dependencies are specified, and should be
fetched automatically:

\begin{itemize}
  \item{Google Guava}
  \item{Apache commons-lang}
  \item{Apache commons-collections}
  \item{Apache commons-configuration}
  \item{JADE}
\end{itemize}

\subsection{From the command line}
To run the simulation from the command line:
\begin{enumerate}
\item{\verb!mvn clean compile exec:java!}
\end{enumerate}

\subsection{Eclipse}
To run the simulation in Eclipse:
\begin{enumerate}
\item{Extract the zip to a folder}
\item{In Eclipse: \verb!File -> Import -> Maven -> Existing Maven Projects!.}
\item{Set Root Directory to be the extracted folder. It should identify the
\verb!pom.xml! file}
\item{Complete the wizard: \verb!Next -> Finish!.}
\item{Load the launch config:
\verb!Run -> Run Configurations -> Maven Build -> mvn_exec_java!}
\item{Click \verb!Run!}
\end{enumerate}

The output is a series of printed statements on \verb!STDOUT! about the state
of the network.  To print this report more frequently (by default it's every
second), change the \verb!runner.report_interval_ms! config var in
\verb!P2PPowerLaw.properties!.

When the simulation is finished, a report will be printed detailing an ordered
list of how many messages each Peer sent. You can customize the halting
conditions in the properties file. By default, the simulation will finish when
either every peer has acquired their wanted files, or 1\@,000\@,000 messages
have been sent. NB: The simulation does not actually quit, owing to
difficulties in cleanly terminating all agents. Therefore, the process will
have to be interrupted to truly stop the simulation.

\section{Deviations and Limitations}
Analysis of the behaviour of the system reveals certain areas which diverge in
some way from the required behaviour.

If a Super Peer is disconnected from the network, then there is no mechanism
for its connected peers to be made aware of this fact.  Nonetheless, once a
Peer has been told that the Super Peer is disconnected (either by a third part
or by the peer periodically Pinging their current connections), the simulation
would correctly handle the situation by adding more connections if necessary.
Therefore, what remains is an extra Behaviour for all peers to periodically
check if their connections are still connected.

Instead of reporting to the super peer all the files it requests, Peers send
search requests individually. The resulting behaviour is equivalent, although
it means for peers with more than 1 Super Peer connection (i.e. Super Peers),
each search request may go to a different connected Super Peer.

One limitation that has surfaced despite following the spec is the potential
for islands to form in the network. Since no guarantee is placed on which
peers connect to which, it's possible for the network to form into two or more
groups.  For example, take the case where Super Peers can have a maximum of 2
peers connected to them. Imagine there are 2 Super peers connected to each
other, leaving one connection space free for both.  This connection space
might be occupied by an Ordinary Peer, for both Super Peers.  This set of 4
peers is now unable to join any other parts of the network.  If one of these
peers is the peer which has a file everyone else wants (in the case of
Scenario 1), the rest of the network will be unable to retrieve this file.
This problem can be made less likely by having a smaller number of Super Peers
in the network (and adjusting the capacity of super peers accordingly).

It is not possible to directly change the bandwidth of individual peers or
groups of peers. This is largely because of difficulties in specifying
individual properties for peers when there are a large number of peers being
generated.  In order to simulate the effects of changing the bandwidth of
peers, you can specify the proportion of peers that should be Super Peers.
This is a probability; no guarantees are given to the actual number of Super
Peers.

\section{Configuration}

The project can be configured in a variety of ways. For example, by changing
the total number of peers or the proportion of Super Peers.

The configuration supplied is set to create 1000 Peers, of which about 30 are
Super Peers. Each Super Peer is able to accept connection requests from up to
60 Ordinary Peers. Ordinary Peers, as per the spec, always connect to 1 Super
Peer.  Super peers on the other hand must connect to 2 other Super Peers. The
Host Cache returns the addresses of 20 Super Peers when asked.

The files shared and wanted by peers are configured to match Scenario 1 in the
spec, where 1 peer has a file that everyone else wants.

For a full list of options and how they change the simulation, see the
commented \verb!P2P_PowerLaw.properties! file. Almost all constants in the
application can be changed in this way.

